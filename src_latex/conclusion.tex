\section{Conclusion}\label{conclusion}

%In this paper, we developed an extension for ESLint in order to extract a set of code smells from five JavaScript projects. Then, we use Cox hazard models to find out whether smelly codes are harmful, and what specific types of code smells make it more defect-prone. We analyze the models to understand which covariates play significant roles in determining defects. We used log-rank test to measure whether a significant statistical difference exists between survival probabilities of smelly codes and non-smelly codes. We also demonstrate the validity and accuracy of our models using the non-proportionality assumption test. Moreover, we designed a survey to have JavaScript practitioners' opinions on the different types of source code practices, i.e., source codes with smells against source codes with the same rationale but without smells.
In this study, we examine the impact of code smells on the fault-proneness {\color{blue} and vulnerability} of JavaScript systems. {\color{blue}Also, we present a survival study of the smells of JavaScript systems.} We present a quantitative study of {\color{blue}fifteen} JavaScript systems that compare the time until a fault occurrence {\color{blue}or a vulnerability appearance} in JavaScript files that contain code smells and files without code smells, {\color{blue}with three different approaches: file grain, line grain, and line grain including dependencies approaches}. {\color{blue}This quantitative study also present some descriptive statistics about the twelve studied smells, as well as their survival by computing their lifetime.} Results show that JavaScript files without code smells have hazard rates {\color{blue}76\%} lower than JavaScript files with code smells {\color{blue}in the file grain study, and 38\% lower than JavaScript files with code smells in the line grain including dependencies study}. In other terms, the survival of JavaScript files against the occurrence of faults increases with time if the files do not contain code smells. We further investigated hazard rates associated with different types of code smells and found that ``Variable Re-assign", ``Assignment in Conditional Statements" {\color{blue}Complex Code} smells have the highest hazard rates. {\color{blue} However, in regards to vulnerability study, we could not say that JavaScript files with code smells are more vulnerable than those without code smells, but we still consider that a better vulnerability database needs to be used in order to get more precise conclusions. Nevertheless, we found that ``Variable Re-assign" and ``This Assign" code smells are more subject to vulnerability than the other smell types. The survival results show us that smells are introduced at the JavaScript files creation most of the time, and a big part of them still survived presently; those smells, and particularly ``Variable Re-assign" which is the most proliferated into the studied systems, have a high chance of surviving a very long time.} In addition, we conducted a survey with 1,484 JavaScript developers, to understand the perception of developers towards our studied code smells, and found that developers consider \emph{Nested Callbacks}, \emph{Variable Re-assign}, \emph{Long Parameter List} to be the most hazardous code smells. JavaScript developers should consider removing \emph{Variable Re-assign} code smells from their systems in priority since this code smell is consistently associated with a high risk of fault, {\color{blue}the highest risk of vulnerability, and because it is the most sizable code smell with a high chance of surviving over time}. They should also prioritize \emph{Assignment in Conditional Statements},\emph{Complex Code}, \emph{This Assign}, \emph{Nested Callbacks}, and \emph{Long Parameter List} code smells for refactoring.

%
%  the occurenand a qualitative study by posting a survey among more than 1000 JavaScript developers, we have made two findings. First, we found that smelly code is more risky than non-smelly code; the risk seems to be system independent. We discovered that on average non-smelly codes have hazard rates 65\% lower than smelly codes. In other terms, the survival of all files against defects increases with time if they do not hold code smells. Second, in our findings we observed that all code smells do not equally affect software maintainability and reliability since variable-reassign (\textsl{var.reassign}) and assignment in condition (\textsl{cond.assign}) code smells have higher hazard rate in comparison to the other types of smells identified by our Cox model and survey.
